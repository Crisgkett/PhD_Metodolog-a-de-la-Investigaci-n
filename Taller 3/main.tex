\documentclass{article}
\usepackage{amsmath}
\usepackage[utf8]{inputenc}
\usepackage{graphicx} % Comandos para manejar imágenes
\graphicspath{ {./images/} } % Carpeta de imágenes
\usepackage[table,xcdraw]{xcolor}
\setlength{\parskip}{2mm} % Espaciado

\usepackage[utf8]{inputenc}
\usepackage{geometry}
    \geometry{left=3cm,right=2cm,top=2cm,bottom=2cm}
%
\usepackage[spanish]{babel}
%
\usepackage[fixlanguage]{babelbib}
    \bibliographystyle{babunsrt}
%

\usepackage{floatrow}
\floatsetup[table]{style=plaintop}

\usepackage{url}

\usepackage[top=2cm, bottom=2.5cm, right=3 cm, left=3 cm]{geometry} % margenes

\usepackage{parskip} % Sangria

\title{Taller 3 A1: Metodología de la Investigación: Buen Gusto}
\author{Cristóbal Galleguillos Ketterer$^{1}$\\
\small{$^{1}$Industrial PhD Program}\\
\small{Pontificia Universidad Católica de Valparaíso}\\
\small{cristobal.galleguillos@pucv.cl}
}
\date{\small{\today}}

\begin{document}

\maketitle

\section{Introducción}

En un interesante artículo periodístico\cite{libertella_no_2020}, el escritor argentino Mauro Libertella plantea una curiosa afirmación “No se puede confiar en una persona a la que no le gusten los Beatles”,  el texto expone una insólita relación entre dos cosas (aparentemente) subjetivas, la percepción de alguien como “confiable” y sus gustos musicales.

A continuación presentaremos, a partir de este ejemplo como podríamos desarrollar el concepto de “Buen Gusto” acotado a cierto tipo de industria cultural.

\section{Sobre el concepto, las dimensiones y los índices}

La industria cultural \cite{ojeda_manifiesto_2019} se compone de una serie de bienes y servicios que están destinados a satisfacer algún tipo de gusto para determinada audiencia, mediante la venta y comercialización de estos propios bienes y servicios.

Así es como asistir a un recital o pagar la entrada para ver “Las Meninas” está incluido dentro de la denominada industria cultural, la pregunta que nos planteamos es: ¿Son esos gustos ponderables, medibles y posibles de ordenar en calidad, de acuerdo a patrones objetivos?

Definiremos “Buen Gusto” a partir de las respuestas a una serie de preguntas con distinto peso, las respuestas están basadas en cuatro dimensiones (Musical, Audiovisual, Literatura, Educacional). A cada una de esas dimensiones se le asignan indicadores definidos de forma arbitraria como:

\begin{itemize}
    \item No se puede confiar en una persona a la que no le guste(n) la música clásica, el Jazz
    \item El cine es Francis Ford Coppola
    \item Yo leo ensayos sobre los distintos significados de Ítaca en la obra de Homero
    \item La Universidad es un edificio viejo que está en Brasil con Av. Argentina
    
\end{itemize}

El cine es Francis Ford Coppola
Yo leo ensayos sobre los distintos significados de Ítaca en la obra de Homero
La Universidad es un edificio viejo que está en Brasil con Av. Argentina

A cada indicador se le asigna un peso y a partir de él se aplican los siguientes índices (se aplica esta encuesta a un grupo hipotético de amigas):

\begin{itemize}
    \item Suma simple: Permite identifica a la o las chicas con más “Buen Gusto”
    \item Media simple: es el promedio de los puntajes obtenidos en cada una de las respuestas, en este caso particular, tiene un valor que es estadísticamente significativo al mismo nivel que la suma simple.
    \item Máximo: Identifica cual es la dimensión en la que la encuestada tiene más “Buen Gusto”
    \item Máximo: Identifica cual es la dimensión en la que la encuestada tiene menos “Buen Gusto” 
\end{itemize}


\section{Comentarios}
 
No podemos dudar del planteamiento de fondo de Libertella, el cual esconde interrogantes tan complejas que hablan del cómo somos a partir de lo que mostramos o lo las cosas que disfrutamos.

La pregunta clave es si, los gustos culturales pueden definir aspectos de una persona como la confianza que podemos tener en ellas o la capacidad de un político para desarrollar sus funciones.

El problema, en este caso particular, es como limitamos o acotamos el número de dimensiones a considerar en el estudio, tomando en cuenta, que la elección podría ser muy pobre o al revés contener demasiada información que no sería útil, en este caso, podríamos hablar de porque se excluyeron dimensiones como las artes plásticas, los viajes o incluso el deporte.

Otro aspecto a considerar es como desarrollar indicadores que no sean son demasiado arbitrarios y que hablen más de autor del estudio que de la población a estudiar. Lo mismo pasa con los ponderadores.

Se puede decir que establecer un juicio acerca de una persona a partir de sus gustos, es muy complejo y quizás carente de rigurosidad científica,  como podríamos valorar que estos gustos estén determinados por diversos orígenes: culturales, emocionales o simplemente por acceso.
 

\nocite{*}
    \bibliography{src/Leiva3A1}

\end{document}
